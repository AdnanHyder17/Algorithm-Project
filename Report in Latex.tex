% Options for packages loaded elsewhere
\PassOptionsToPackage{unicode}{hyperref}
\PassOptionsToPackage{hyphens}{url}
%
\documentclass[
]{article}
\usepackage{amsmath,amssymb}
\usepackage{iftex}
\ifPDFTeX
  \usepackage[T1]{fontenc}
  \usepackage[utf8]{inputenc}
  \usepackage{textcomp} % provide euro and other symbols
\else % if luatex or xetex
  \usepackage{unicode-math} % this also loads fontspec
  \defaultfontfeatures{Scale=MatchLowercase}
  \defaultfontfeatures[\rmfamily]{Ligatures=TeX,Scale=1}
\fi
\usepackage{lmodern}
\ifPDFTeX\else
  % xetex/luatex font selection
\fi
% Use upquote if available, for straight quotes in verbatim environments
\IfFileExists{upquote.sty}{\usepackage{upquote}}{}
\IfFileExists{microtype.sty}{% use microtype if available
  \usepackage[]{microtype}
  \UseMicrotypeSet[protrusion]{basicmath} % disable protrusion for tt fonts
}{}
\makeatletter
\@ifundefined{KOMAClassName}{% if non-KOMA class
  \IfFileExists{parskip.sty}{%
    \usepackage{parskip}
  }{% else
    \setlength{\parindent}{0pt}
    \setlength{\parskip}{6pt plus 2pt minus 1pt}}
}{% if KOMA class
  \KOMAoptions{parskip=half}}
\makeatother
\usepackage{xcolor}
\usepackage{graphicx}
\makeatletter
\def\maxwidth{\ifdim\Gin@nat@width>\linewidth\linewidth\else\Gin@nat@width\fi}
\def\maxheight{\ifdim\Gin@nat@height>\textheight\textheight\else\Gin@nat@height\fi}
\makeatother
% Scale images if necessary, so that they will not overflow the page
% margins by default, and it is still possible to overwrite the defaults
% using explicit options in \includegraphics[width, height, ...]{}
\setkeys{Gin}{width=\maxwidth,height=\maxheight,keepaspectratio}
% Set default figure placement to htbp
\makeatletter
\def\fps@figure{htbp}
\makeatother
\ifLuaTeX
  \usepackage{luacolor}
  \usepackage[soul]{lua-ul}
\else
  \usepackage{soul}
\fi
\setlength{\emergencystretch}{3em} % prevent overfull lines
\providecommand{\tightlist}{%
  \setlength{\itemsep}{0pt}\setlength{\parskip}{0pt}}
\setcounter{secnumdepth}{-\maxdimen} % remove section numbering
\ifLuaTeX
  \usepackage{selnolig}  % disable illegal ligatures
\fi
\IfFileExists{bookmark.sty}{\usepackage{bookmark}}{\usepackage{hyperref}}
\IfFileExists{xurl.sty}{\usepackage{xurl}}{} % add URL line breaks if available
\urlstyle{same}
\hypersetup{
  pdftitle={Report: Geometric Algorithms},
  hidelinks,
  pdfcreator={LaTeX via pandoc}}

\title{Report: Geometric Algorithms}
\author{}
\date{}

\begin{document}
\maketitle

\subsubsection{Members:}\label{members}

\begin{itemize}
\item
  Adnan Haider (21K-3445)
\item
  Hamza Raza (21K-4699)
\end{itemize}

\subsection{Abstract}\label{abstract}

The Geometric Algorithms Demo is a visually appealing application
developed using Python\textquotesingle s Tkinter framework. It provides
an interactive platform for users to comprehend and experiment with
diverse geometric techniques, such as computing convex hulls and
detecting line segment intersections. This paper offers a summary of the
programming structure, experimental setup, and application outcomes.

\subsubsection{Introduction}\label{introduction}

Geometric algorithms play a crucial role in computer science and
computational geometry. The Geometric Algorithms Demo aims to provide a
user-friendly platform for understanding and experimenting with key
geometric algorithms. The implemented algorithms include convex hull
computation using Jarvis March, Graham Scan, Quick Elimination, Brute
Force, and Monotone Algorithm. Additionally, the
application enables users to input points manually or load them from a
text file, check for line segment intersections, and visualize the
results.

Tinker Library

The application\textquotesingle s core functionality is encapsulated
within a class called GeometricAlgorithmsDemo, which is initialized with
the root window generated by Tkinter\textquotesingle s Tk class. The
class has several methods and attributes, including:

- \_\_init\_\_: This method initializes the GeometricAlgorithmsDemo
object and sets up the GUI. It creates a canvas widget, buttons for
resetting and switching between algorithms, and draws the grid axes. It
also binds event handlers to mouse clicks and key presses.

- reset: This method resets the GUI by deleting all objects from the
canvas widget and initializing lists of points and lines. It also
redraws the grid axes.

- switch\_algorithm: This method switches between the convex hull
computation and line segment intersection detection algorithms by
updating the displayed text, drawing functions, and event handlers.

- compute\_convex\_hull: This method calculates the convex hull of a set
of points using an efficient algorithm and draws it on the canvas widget
using the Polygon class from Tkinter\textquotesingle s turtle module.

- detect\_intersections: This method determines whether two line
segments intersect using a simple algorithm based on their
endpoints\textquotesingle{} coordinates and draws them on the canvas
widget using the Line class from Tkinter\textquotesingle s turtle
module.

- draw\_points: This method draws a set of points on the canvas widget
using the Circle class from Tkinter\textquotesingle s turtle module.

- draw\_lines: This method draws a set of lines connecting pairs of
points on the canvas widget using the Line class from
Tkinter\textquotesingle s turtle module.

- draw\_convex\_hull: This method draws the convex hull of a set of
points on the canvas widget using the Polygon class from
Tkinter\textquotesingle s turtle module.

- draw\_intersections: This method draws the intersections between two
line segments on the canvas widget using the Circle class from
Tkinter\textquotesingle s turtle module.

The application uses several built-in functions from
Python\textquotesingle s math module, such as dot product, cross
product, and distance calculation, to implement its algorithms
accurately and efficiently. The user can interact with these algorithms
by clicking on points to add them to sets, pressing keys to switch
between algorithms or reset them, and observing their results in
real-time on the canvas widget.

\subsubsection{Programming Design}\label{programming-design}

The main components of the design include:

\begin{itemize}
\item
  \textbf{Canvas}: A Tkinter canvas widget is used for drawing points,
  line segments, and convex hulls.
\item
  \textbf{Result Display}: A label displays the results of intersection
  checks and convex hull computations.
\item
  \textbf{Animation}: Convex hulls are visualized through animation on
  the canvas.
\end{itemize}

\subsubsection{Experimental Setup}\label{experimental-setup}

The application was developed using Python and the following libraries:

\begin{itemize}
\item
  Tkinter for the graphical user interface.
\item
  Convex hull algorithms (Jarvis March, Graham Scan, Quick Elimination,
  Brute Force, Monotone Algorithm) are implemented in
  separate modules.
\end{itemize}

\subsubsection{Results and Discussion}\label{results-and-discussion}

The application was tested with various scenarios to demonstrate its
functionality. Screenshots and discussions for key features are provided
below.

\subsubsection{Line Segment Intersection
Check}\label{line-segment-intersection-check}

\includegraphics[width=4.43449in,height=4.25672in]{image6.png}

\subsubsection{Convex Hull Computation}\label{convex-hull-computation}

\includegraphics[width=4.71875in,height=4.0836in]{image1.png}

\includegraphics[width=5.89063in,height=4.51716in]{image2.png}

\includegraphics[width=5.57813in,height=4.24282in]{image3.png}

\includegraphics[width=5.3125in,height=4.55729in]{image4.png}

\includegraphics[width=5.09896in,height=4.11458in]{image5.png}

\subsubsection{Execution Time
Comparison}\label{execution-time-comparison}

\begin{itemize}
\item
  Jarvis March, Time = 0.001041 seconds
\item
  Graham Scan, Time = 0.0005188 seconds
\item
  Quick Elimination, Time = 0.000731 seconds
\item
  Brute Force, Time = 0.001506 seconds
\item
  Monotone Algorithm, Time = 0.00074339 seconds
\end{itemize}

\subsubsection{Conclusion}\label{conclusion}

The Geometric Algorithms Demo successfully implements key geometric
algorithms, providing a hands-on learning experience for users. The
application\textquotesingle s interactive features make it a valuable
tool for educational purposes and algorithm exploration. Further
improvements and enhancements could include additional algorithms and
optimizations.

\subsubsection{References}\label{references}

\begin{itemize}
\item
  Tkinter Documentation:
  \href{https://docs.python.org/3/library/tkinter.html}{\ul{https://docs.python.org/3/library/tkinter.html}}
\item
  Convex Hull Algorithms: \emph{(Include relevant references or
  documentation for the convex hull algorithms used.)}
\end{itemize}

\end{document}
